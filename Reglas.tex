\documentclass[10pt,letterpaper]{article}
\usepackage[latin1]{inputenc}
\usepackage[spanish]{babel}
\usepackage{amsmath}
\usepackage{amsfonts}
\usepackage{amssymb}
\usepackage{graphicx}
\usepackage[left=2cm,right=2cm,top=2cm,bottom=2cm]{geometry}
\author{Andres Cuervo y Samuel Restrepo}
\title{Reglas del buscaminas}
\date{Octubre 2019}
\begin{document}
\maketitle

\section{Cada casilla, sin importar la elegida, retornara el numero 1 siempre, indicando que alrededor del cuadro elegido se encuentra una bomba.
}
$Casilla escogida \rightarrow   {Posibles casillas con bomba}$\\
$C1 \rightarrow{C2 V C4 V C5}$ \\
$C2 \rightarrow{C1 V C3 V C4 V C5 V C6}$\\
$C3 \rightarrow {C2 V C5 V C6}$\\
$C4 \rightarrow {C1 V C2 V C5 V C7 V C8}$\\
$C5 \rightarrow {C1 V C2 V C3 V C4 V C6 V C7 V C8 V C9}$\\
$C6 \rightarrow {C2 V C3 V C5 V C8 V C9}$ \\
$C7 \rightarrow {C4 V C5 V C8}$\\
$C8 \rightarrow {C4 V C5 V C6 V C7 V C9}$\\
$C9 \rightarrow {C5 V C6 V C8}$\\
\section{Se tendran que elegir 2 casillas consecutivas. Se supondra igualmente que la
segunda casilla tampoco corresponde a una bomba}
Ci Y Cj escogidas = La interseccion de los conjuntos solucion de cada casilla\\
$C1 AND C2 \rightarrow {C4 V C5}$\\
$C1 AND C4 \rightarrow {C2 V C5}$\\
$C2 AND C3 \rightarrow {C5 V C6}$\\
$C2 AND C5 \rightarrow {C1 V C3 V C4 V C6}$\\
$C3 AND C6 \rightarrow {C2 V C5}$\\
$C4 AND C5 \rightarrow {C1 V C2 V C7 V C8}$\\
$C4 AND C7 \rightarrow {C5 V C8}$\\
$C5 AND C6 \rightarrow {C2 V C3 V C8 V C9}$\\
$C5 AND C8 \rightarrow {C4 V C6 C7 V C9}$\\
$C6 AND C9 \rightarrow {C5 V C8}$\\
$C7 AND C8 \rightarrow {C4 V C5}$\\
$C8 AND C9 \rightarrow {C5 V C6}$\\
\section{Se tomara un cuadro 3 POR 3.}
Al tomar un cuadro de 3x3 entonces habran 9 casillas: C1,C2,C3,C4,C5,C6,C7,C8 Y C9.\\

\end{document}
