\documentclass[10pt,a4paper]{article}
\usepackage[latin1]{inputenc}
\usepackage[T1]{fontenc}
\usepackage{amsmath}
\usepackage{amsfonts}
\usepackage[spanish]{babel}
\usepackage{amssymb}
\usepackage{graphicx}
\author{Andr�s Cuervo y Samuel Restrepo}
\title{Buscaminas}
\begin{document}
	\title{Buscaminas}
	\maketitle
	\section{La situaci�n global}
	
	Queremos representar un caso del juego buscaminas de un tablero 3x3 en el cual cada casilla que se oprima va a tener el n�mero 1, es decir, va a tener una bomba alrededor de esta.
	\section{El proposito}
	
	El prop�sito de representar este caso del juego buscaminas es para aprender a hacer uso del juego para que una persona que no conozca el juego se adapte f�cilmente. Adem�s, funciona para plantear situaciones de c�mo se debe ganar con base en las elecciones que uno toma en el transcurso del juego. Por lo tanto, los dos prop�sitos son ense�ar y plantear las situaciones de c�mo resolverlo desde la l�gica.
	
	\section{Reglas}
	
	1.Suponer que cada casilla que se tome va a retornar n�mero 1 y por ende no va a haber bomba en esa.\\
	2.Tomar dos casillas consecutivas.\\
	
	
\end{document}