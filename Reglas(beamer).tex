\documentclass{beamer}
\usetheme{CambridgeUS}
\usepackage[T1]{fontenc}
\usepackage[utf8]{inputenc}
\usepackage[spanish,es-tabla]{babel}
\usepackage{amsmath}
\usepackage{amssymb,amsfonts,latexsym,cancel}
\usepackage{float}
\usepackage{graphicx}
\usepackage{epstopdf}
\usepackage{subfigure}

\title{Reglas del buscaminas}
\author{Samuel Restrepo y Andrés Cuervo}
\date{Octubre 2019} 

\begin{document}

\begin{frame}
\titlepage
\end{frame}

\begin{frame}
\frametitle{Regla 1}
\framesubtitle{La primera casilla sin importar la elegida, retornara el numero 1 siempre, indicando que alrededor del cuadro elegido se encuentra una bomba.
}

Representación :\\
Ci = la casilla i es una bomba\\
$\neg{Ci}$ = la casilla i no es una bomba\\

\end{frame}

\begin{frame}
\frametitle{Representación lógica de la regla 1}
$Casilla escogida \rightarrow   {Posibles casillas con bomba}$\\
$\neg{C1} \rightarrow{(C2 \wedge{\neg{C4}} \wedge{\neg{C5}})\lor{(C4 \wedge{\neg{C2}} \wedge{\neg{C5}})}\lor{(C5 \wedge{\neg{C2}} \wedge{\neg{C4}})}}$\\
$\neg{C2} \rightarrow{(C1 \wedge{\neg{C3}} \wedge{\neg{C4}} \wedge{\neg{C5}} \wedge{\neg{C6}}})$\\
$\lor{(C3 \wedge{\neg{C1}} \wedge{\neg{C4}} \wedge{\neg{C5}}\wedge{\neg{C6}})}\lor{(C4 \wedge{\neg{C1}} \wedge{\neg{C3}} \wedge{\neg{C5}} \wedge{\neg{C6}})}$ \\
$\lor{}$ .....\\
$\neg{C3} \rightarrow{(C2 \wedge{\neg{C5}} \wedge{\neg{C6}})\lor{(C5 \wedge{\neg{C2}} \wedge{\neg{C6}})}\lor{(C6 \wedge{\neg{C2}} \wedge{\neg{C5}})}}$\\



\end{frame}



\begin{frame}
\frametitle{Regla 2}
\framesubtitle{Se tendran que elegir 2 casillas consecutivas. Se supondra igualmente que la segunda casilla tampoco corresponde a una bomba}

$\neg{C1} \wedge{\neg{C2}} \rightarrow {(C4 \wedge{\neg{C3}} \wedge{\neg{C5}} \wedge{\neg{C6})\lor{(C5 \wedge{\neg{C3}} \wedge{\neg{C4}} \wedge{\neg{C6})}}}}$\\
$\neg{C1} \wedge{\neg{C4}} \rightarrow {(C2 \wedge{\neg{C5}} \wedge{\neg{C7}} \wedge{\neg{C8})\lor{(C5 \wedge{\neg{C2}} \wedge{\neg{C7}} \wedge{\neg{C8})}}}}$\\
$\neg{C2} \wedge{\neg{C3}} \rightarrow {(C5 \wedge{\neg{C1}} \wedge{\neg{C4}} \wedge{\neg{C6})\lor{(C6 \wedge{\neg{C1}} \wedge{\neg{C4}} \wedge{\neg{C5})}}}}$\\
Y así con todas las posibles uniones de casillas consecutivas que son:\\
$C2 \wedge{ C5} $\\
$C3 \wedge{ C6} $\\
$C4 \wedge{ C5} $\\
$C4 \wedge{ C7} $\\
$C5 \wedge{ C6} $\\
$C5 \wedge{ C8} $\\
$C6 \wedge{ C9} $\\
$C7 \wedge{ C8} $\\
$C8 \wedge{ C9} $\\
\end{frame}



\begin{frame}
\frametitle{Regla 3}
\framesubtitle{Se tomara un cuadro 3 POR 3}
Al tomar un cuadro de 3x3 entonces habran 9 casillas: C1,C2,C3,C4,C5,C6,C7,C8 Y C9.\\
\end{frame}

\end{document}

\begin{comment}
$\neg{C2} \rightarrow{C1 V C3 V C4 V C5 V C6}$\\
$\neg{C3} \rightarrow {C2 V C5 V C6}$\\
$\neg{C4} \rightarrow {C1 V C2 V C5 V C7 V C8}$\\
$\neg{C5} \rightarrow {C1 V C2 V C3 V C4 V C6 V C7 V C8 V C9}$\\
$\neg{C6} \rightarrow {C2 V C3 V C5 V C8 V C9}$ \\
$\neg{C7} \rightarrow {C4 V C5 V C8}$\\
$\neg{C8} \rightarrow {C4 V C5 V C6 V C7 V C9}$\\
$\neg{C9} \rightarrow {C5 V C6 V C8}$\\
\end{comment}
